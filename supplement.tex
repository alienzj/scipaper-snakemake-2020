\documentclass{scrartcl}
\usepackage[utf8]{inputenc}
\usepackage{amsmath}
\usepackage{graphicx}
\usepackage{listings}
\usepackage[hidelinks]{hyperref}
\usepackage{enumitem}
\setlist[enumerate]{itemsep=0mm}
\setlist[itemize]{itemsep=0mm}

% Cross referencing external documents.
\usepackage{xr}
% Command for registering file dependencies with latexmk.
\makeatletter
\newcommand*{\addFileDependency}[1]{% argument=file name and extension
  \typeout{(#1)}
  \@addtofilelist{#1}
  \IfFileExists{#1}{}{\typeout{No file #1.}}
}
\makeatother

\externaldocument[m-]{main}
\addFileDependency{main.tex}
\addFileDependency{main.aux}

\renewcommand{\thesection}{S\arabic{section}}
\renewcommand{\thefigure}{S\arabic{figure}}

\newcommand{\image}[1]{\centering\includegraphics[width=\textwidth]{#1}}
\let\plainurl\url
\renewcommand{\url}[1]{\protect\plainurl{#1}}

\usepackage[backend=bibtex,maxbibnames=999]{biblatex}
\addbibresource{literature.bib}

\begin{document}

\title{Supplement: Sustainable data analysis with Snakemake}
\maketitle

\section{Readability}\label{sec:readability}
Statements in Snakemake workflow definitions fall into seven categories:
\begin{enumerate}
	\item a natural language word, followed by a colon (e.g.~\lstinline!input:! and~\lstinline!output:!),
	\item the word ``rule'', followed by a name and a colon (e.g. \lstinline!rule\ convert\_to\_pdf:!),
	\item a quoted filename pattern (e.g. \lstinline!"{prefix}.pdf"!),
	\item a quoted shell command,
	\item a quoted wrapper identifier,
	\item a quoted container URL
	\item a Python statement.
\end{enumerate}

Below, we list the rationale of our assessment for each category in \autoref{m-fig:example}: 

\begin{enumerate}
	\item The natural language word is either understandable with general education~(e.g.\ \lstinline!input:! and \lstinline!output:!) or technical knowledge (\lstinline!container:! or~\lstinline!conda:!).
	      The colon straightforwardly shows that the content follows next.
	      Only for the wrapper directive (\lstinline!wrapper:!) one needs to have the Snakemake specific knowledge that it is possible to refer to publicly available tool wrappers.
	\item
	      The word rule is understandable with general education, and when
	      carefully choosing rule names, at most domain knowledge is needed for
	      understanding such statements.
	\item
	      Filename patterns can mostly be understood with domain knowledge,
	      since the file extensions should tell the expert what kind of content
	      will be used or created.
	      We hypothesize that wildcard definitions (e.g.~\lstinline!{country}!) are straightforwardly understandable as a placeholder.
	\item
	      Shell commands will usually need domain and technical knowledge for
	      reception.
	\item
	      Wrapper identifiers can be understood with Snakemake knowledge only,
	      since one needs to know about the central tool wrapper repository of
	      Snakemake.
	      Nevertheless, with only domain knowledge one can at least conclude that the mentioned tool (last part of the wrapper ID) will be used in the wrapper.
	\item
	      A container URL will usually be understandable with technical
	      knowledge.
	\item
	      Python statements will either need technical knowledge or Snakemake
	      knowledge (when using the Snakemake API, as it happens here with the
	      expand command, which allows to aggregate over a combination of
	      wildcard values).
\end{enumerate}

\section{Design patterns}\label{sec:design-patterns}

\autoref{fig:design-patterns} shows advanced design patterns which are less common but useful in certain situations.
Below, we explain each example in detail.

\begin{figure}
	\image{design-patterns.pdf}
	\caption{Additional design patterns for Snakemake workflows.
		For brevity only rule properties that are necessary to understand each example are shown (e.g. omitting log directives and shell commands).
		(a) scatter/gather process, (b) streaming, (c) non-file parameters, (d) sample sheet based configuration, (e) conditional execution, (f) benchmarking.
		See \autoref{sec:design-patterns} for details.
	}\label{fig:design-patterns}
\end{figure}

\paragraph{Scatter/gather processes (\autoref{fig:design-patterns}a).}
Snakemakes ability to emply arbitrary Python code for defining a rule's input and output files already enables any kind of scattering, gathering, and aggregations in workflows.
Nevertheless, it can be more readable and scalable to use Snakemake's explicit support for scatter/gather processes.
A Snakemake workflow can have any number of such processes, each of which has a name (here \lstinline!someprocess!).
In this example, the rule \lstinline!scatter! splits an data into $n$ items; the rule \lstinline!step2! is applied to each item; the rule \lstinline!gather! aggregates over the outputs of \lstinline!step2! for each item.
Thereby, $n$ is defined via the \lstinline!scattergather! directive at the beginning, which sets $n$ for each scatter/gather process in the workflow.
In addition, $n$ can be set via the command line via the flag \lstinline!--set-scatter!.
For example, here, we could set the number of scatter items to 16 by specifying \lstinline!--set-scatter someprocess=16!.

\paragraph{Streaming (\autoref{fig:design-patterns}b).}
Snakemake allows to stream output between jobs, instead of writing it to disk (see \autoref{m-sec:streaming} in the main document).
Here, the output of rule \lstinline!step1! and \lstinline!step2! is streamed into rule \lstinline!step3!.

\paragraph{Non-file parameters (\autoref{fig:design-patterns}c).}
Data analysis steps can need additional non-file input in the form of parameters, that are for example obtained from the workflow configuration (see \autoref{m-sec:automation} in the main document).
Both input files and such non-file parameters can optionally be defined via a Python function, which is evaluated for each job, when wildcard values are known.
In this example, we define a lambda expression (an anonymous function in Python), that retrieves a threshold depending on the value of the wildcard sample (\lstinline!w.sample!).
Wildcard values are passed as the first positional argument to such functions (here \lstinline!w!).

\paragraph{Sample sheet based configuration (\autoref{fig:design-patterns}d).}
Often, scientific experiments entail multiple samples, for which meta-information is known (e.g. gender, tissue etc. in biomedicine).
Portable encapsulated projects (PEPs, \url{https://pep.databio.org}) are an approach to standardize such information and provide them in a shareable format.
Snakemake workflows can be directly integrated with PEPs, thereby allowing to configure them via meta-information that is contained in the sample sheets defined by the PEP.
Here, a pepfile is defined, followed by an aggregation over all samples defined in the contained sample sheet.

\paragraph{Conditional execution (\autoref{fig:design-patterns}e).}
By default, Snakemake determines the entire DAG of jobs upfront, before the first job is executed.
However, sometimes the analysis path that shall be taken depends on some intermediate results.
For example, this is the case when filtering samples based on quality control criteria.
At the beginning of the data analysis, some quality control (QC) step is performed, which yields QC values for each sample.
The actual analysis that shall happen afterwards might be only suitable for samples that pass the QC.
Hence, one might have to filter out samples that do not pass the QC.
Since the QC is an intermediate result of the same data analysis, it can be necessary to determine the part of the DAG that comes downstream of the QC only after QC has been finalized.
Of course, one option is to separate QC and the actual analysis into two workflows, or defining a separate target rule for QC, such that it can be manually completed upfront, before the actualy analalysis is started.
Alternatively, if QC shall happen automatically as part of the whole workflow, one can make use of Snakemake's conditional execution capabilities.
In the example, we define that the \lstinline!qc! rule shall be a so-called \lstinline!checkpoint!.
Rules can depend on such checkpoints by obtaining their output from a global \lstinline!checkpoints! object (line 2), that is accessed inside of a function, which is passed to the \lstinline!input! directive of the rule (line 11).
This function is re-evaluated after the checkpoint has been executed (and its output files are present), thereby allowing to inspect the content of the checkoint's output files, and decide about the input files based on that.
In this example, the checkpoint rule \lstinline!qc! creates a TSV file, which the function loads, in order to extract only those samples for which the column \lstinline!"some-value"! contains a value greater than $90$ (line 6).
Only for those samples, the file \lstinline!"results/processed/{sample}.txt"! is requested, which is then generated by applying the rule \lstinline!process! for each of these samples.

\paragraph{Benchmarking (\autoref{fig:design-patterns}f)}
Sometimes, a data analysis entails the benchmarking of certain tools in terms of runtime, CPU, and memory consumption.
Snakemake directly supports such benchmarking by defining a \lstinline!benchmark! directive in a rule (line 7).
This directive takes a path to a TSV file.
Upon execution of a job spawned from such a rule, Snakemake will constantly measure CPU and memory consumption, and store averaged results together with runtime information into the given TSV file.
Benchmark files can be input to other rules, for example in order to generate plots or summary statistics.

\printbibliography
\end{document}
