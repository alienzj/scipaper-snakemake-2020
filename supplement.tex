\documentclass{scrartcl}
\usepackage[utf8]{inputenc}
\usepackage{amsmath}
\usepackage{graphicx}
\usepackage{listings}
\usepackage[hidelinks]{hyperref}
\usepackage{xr}
\externaldocument[m-]{main.tex}

\renewcommand{\thesection}{S\arabic{section}}
\renewcommand{\thefigure}{S\arabic{figure}}

\usepackage[backend=bibtex,maxbibnames=999]{biblatex}
\addbibresource{literature.bib}

\begin{document}

\title{Supplement: Sustainable data analysis with Snakemake}
\maketitle

\section{Readability}\label{sec:readability}
Statements in Snakemake workflow definitions fall into seven categories:
\begin{enumerate}
	\item a natural language word, followed by a colon (e.g.~\lstinline!input:! and~\lstinline!output:!),
	\item the word ``rule'', followed by a name and a colon (e.g. \lstinline!rule\ convert\_to\_pdf:!),
	\item a quoted filename pattern (e.g. \lstinline!"{prefix}.pdf"!),
	\item a quoted shell command,
	\item a quoted wrapper identifier,
	\item a quoted container URL
	\item a Python statement.
\end{enumerate}

Below, we list the rationale of our assessment for each category in \autoref{m-fig:example}: 

\begin{enumerate}
	\item The natural language word is either understandable with general education~(e.g.\ \lstinline!input:! and \lstinline!output:!) or technical knowledge (\lstinline!container:! or~\lstinline!conda:!).
	      The colon straightforwardly shows that the content follows next.
	      Only for the wrapper directive (\lstinline!wrapper:!) one needs to have the Snakemake specific knowledge that it is possible to refer to publicly available tool wrappers.
	\item
	      The word rule is understandable with general education, and when
	      carefully choosing rule names, at most domain knowledge is needed for
	      understanding such statements.
	\item
	      Filename patterns can mostly be understood with domain knowledge,
	      since the file extensions should tell the expert what kind of content
	      will be used or created.
	      We hypothesize that wildcard definitions (e.g.~\lstinline!{country}!) are straightforwardly understandable as a placeholder.
	\item
	      Shell commands will usually need domain and technical knowledge for
	      reception.
	\item
	      Wrapper identifiers can be understood with Snakemake knowledge only,
	      since one needs to know about the central tool wrapper repository of
	      Snakemake.
	      Nevertheless, with only domain knowledge one can at least conclude that the mentioned tool (last part of the wrapper ID) will be used in the wrapper.
	\item
	      A container URL will usually be understandable with technical
	      knowledge.
	\item
	      Python statements will either need technical knowledge or Snakemake
	      knowledge (when using the Snakemake API, as it happens here with the
	      expand command, which allows to aggregate over a combination of
	      wildcard values).
\end{enumerate}

\printbibliography
\end{document}
